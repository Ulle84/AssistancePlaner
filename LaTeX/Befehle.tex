% Eigene Befehle und typographische Auszeichnungen f�r diese

% einfaches Wechseln der Schrift, z.B.: \changefont{cmss}{sbc}{n}
\newcommand{\changefont}[3]{\fontfamily{#1} \fontseries{#2} \fontshape{#3} \selectfont}

% Abk�rzungen mit korrektem Leerraum 
\newcommand{\ua}{\mbox{u.\,a.\ }}
\newcommand{\zB}{\mbox{z.\,B.\ }}
\newcommand{\dahe}{\mbox{d.\,h.\ }}
\newcommand{\Vgl}{Vgl.\ }
\newcommand{\bzw}{bzw.\ }
\newcommand{\evtl}{evtl.\ }

\newcommand{\abbildung}[1]{Abbildung~\ref{fig:#1}}
\newcommand{\gleichung}[1]{Gleichung~\ref{eq:#1}}
\newcommand{\tabelle}[1]{Tabelle~\ref{tab:#1}}
\newcommand{\codelisting}[1]{Listing~\ref{lst:#1}}

\newcommand{\bs}{$\backslash$}

% erzeugt ein Listenelement mit fetter �berschrift 
\newcommand{\itemd}[2]{\item{\textbf{#1}}\\{#2}}

% einige Befehle zum Zitieren --------------------------------------------------
\newcommand{\Zitat}[2][\empty]{\ifthenelse{\equal{#1}{\empty}}{\citep{#2}}{\citep[#1]{#2}}}

% zum Ausgeben von Autoren
\newcommand{\AutorName}[1]{\textsc{#1}}
\newcommand{\Autor}[1]{\AutorName{\citeauthor{#1}}}

% Links mit Zuletzt angerufen am in Fu�note---
%\Link{URL}{Beschreibung}{Datum}
\newcommand{\Link}[3]{\href{#1}{#2}\footnote{\url{#1} - zuletzt abgerufen am #3}}
\newcommand{\LinkWithoutDate}[2]{\href{#1}{#2}\footnote{\url{#1}}}

% neue Abkuerzung - z. B. \NewAbb{UML}{Unified Modling Language}
\newcommand{\NewAbb}[2]{\textit{#2} (\textbf{#1})\nomenclature{#1}{#2}}

% Notizen
\newcommand{\Note}[1]{\par\noindent\colorbox{hellrot}{\parbox{\dimexpr\textwidth-2\fboxsep\relax}{#1}}}

% verschiedene Befehle um W�rter semantisch auszuzeichnen ----------------------
\newcommand{\NeuerBegriff}[1]{(\textbf{#1})}
\newcommand{\Fachbegriff}[1]{\textit{#1}}

\newcommand{\file}[1]{\texttt{#1}}
\newcommand{\folder}[1]{\texttt{#1}}
\newcommand{\button}[1]{Button \enquote{#1}}

\newcommand{\Eingabe}[1]{\texttt{#1}}
\newcommand{\Code}[1]{\lstinline[language=Java, showstringspaces=false,basicstyle=\ttfamily\color{black}\small,tabsize=2]$#1$}
\newcommand{\CodeX}[1]{\lstinline[language=XML, showstringspaces=false,basicstyle=\ttfamily\color{black}\small,tabsize=2]$#1$}

\newcommand{\Datei}[1]{\texttt{#1}}

\newcommand{\Datentyp}[1]{\textsf{#1}}
\newcommand{\XMLElement}[1]{\textsf{#1}}
\newcommand{\Webservice}[1]{\textsf{#1}}