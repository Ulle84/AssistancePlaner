\documentclass[12pt,a4paper,titlepage]{article} % article scrreprt (KOMA Script Report)

%Kurz-Titel: Assistenzplaner
%Lang-Titel: Assistenzplaner - Serverbasierte Software zur Koordinierung der Assistenten für behinderte Menschen

%\linespread{2.00} % Zeilenabstand
										
%%%%% INPUT %%%%%
\usepackage[utf8]{inputenc}
\usepackage{ngerman}
\usepackage[pdftex]{graphicx} %png und pdf kann eingefügt werden
%\usepackage{graphics}

% für Literatur
\usepackage[ngerman]{babel}
\usepackage{babelbib}

% für \newboolean{allpages} \setboolean{allpages}{false} --> \ifallpages ... \fi
\usepackage{ifthen}

%\usepackage{listings}
%\usepackage{amssymb}
%\usepackage{rotating}
\usepackage{amsmath}
\usepackage{subfigure}
\usepackage{rotating}		%um Tabellen zu drehen

% Header
\usepackage{fancyhdr}
\pagestyle{fancy}
\fancyhf{}	%Löschen der Vorbelegung
\fancyhead[R]{\thepage}
\fancyhead[L]{\leftmark}
%\fancyhead[R]{\leftmark \hspace{10mm} \thepage}
%\fancyfoot[L]{\footnotesize{Nichtlineare Bildsensoren}}
%\fancyfoot[C]{\thepage}
%\fancyfoot[R]{\footnotesize{Ulrich Belitz}}

\title{Diplomarbeit: Nichtlineare Bildsensoren}
\author{Ulrich Belitz}

%Paket für spezielle PDF features.
\usepackage[%
	pdftitle={Diplomarbeit: Nichtlineare Bildsensoren},% Titel des PDF Dokuments.
	pdfauthor={Ulrich Belitz},               % Autor des PDF Dokuments.
	pdfsubject={Nichtlineare Bildsensoren}, % Thema des PDF Dokuments.
	pdfcreator={TeXnicCenter and MikTeX},    % Erzeuger des PDF Dokuments.
	%pdfkeywords={}                          % Schlüsselwörter für das PDF.
	% (Diese werden von Suchmaschinen auch für PDF Dokumente indexiert.)
	pdfpagemode=UseOutlines,   % Inhaltsverzeichnis anzeigen beim Öffnen
	pdfdisplaydoctitle=false,  % Dokumenttitel statt Dateiname anzeigen.
	pdflang=de                 % Sprache des Dokuments.
]{hyperref}

%Anführungszeichen
\usepackage[babel,german=quotes]{csquotes}	%\enquote{TEXT}

%Farben
\usepackage{color}
\definecolor{LinkColor}{rgb}{0, 0, 0}
\definecolor{LinkColor1}{rgb}{0,0,0}
\definecolor{LinkColor2}{rgb}{0,0.5,0}

%Farbeinstellungen für die Links im PDF Dokument.

\hypersetup{%
	colorlinks=true,%        Aktivieren von farbigen Links im Dokument (keine Rahmen)
	linkcolor=LinkColor,%    Farbe festlegen.
	citecolor=LinkColor2,%   	Farbe festlegen.
	filecolor=LinkColor1,%    Farbe festlegen.
	menucolor=LinkColor,%    Farbe festlegen.
	urlcolor=LinkColor1,%     Farbe von URL's im Dokument.
	bookmarksnumbered=true,%  Überschriftsnummerierung im PDF Inhalt anzeigen.
	pdfstartview=FitH%        FitH || FitV  -  Horizontalen Platz ausnutzen
}

% manuelle Silbentrennung
\hyphenation{Knick-stel-le}
\hyphenation{Knick-stel-len}

\newlength{\plotBreite} \setlength{\plotBreite}{0.9\textwidth}

\usepackage{listings,xcolor}
\usepackage{inconsolata}

\definecolor{dkgreen}{rgb}{0,.6,0}
\definecolor{dkblue}{rgb}{0,0,.6}
\definecolor{dkyellow}{cmyk}{0,0,.8,.3}

\lstset{
  language        = php,
  basicstyle      = \small\ttfamily,
  keywordstyle    = \color{dkblue},
  stringstyle     = \color{red},
  identifierstyle = \color{dkgreen},
  commentstyle    = \color{gray},
  emph            =[1]{php},
  emphstyle       =[1]\color{black},
  emph            =[2]{if,and,or,else},
  emphstyle       =[2]\color{dkyellow}}

\begin{document}
\pagenumbering{Roman}
\begin{titlepage}
%\includegraphics{img/sickLogo.pdf}
\hspace{35mm} % per Hand angepasst, muss geändert werden, falls sich Dokumentabmaße ändern!
%\includegraphics{img/LogoFBMN}
\begin{flushright}
	\scriptsize
	Studiengang Optotechnik und Bildverarbeitung\\[4cm]
\end{flushright}
%\vspace*{4cm}
\begin{center}
	
%	zur Erlangung des akademischen Grades eines \\[5mm]
%		\textbf{Diplomingenieur (FH)\\[1cm]}
	\textbf{\Huge{Nichtlineare Bildsensoren}}\\[4mm]%\\[10mm]}
	\textbf{\large{(Non-linear Imagers)}}\\[10mm]
	\textbf{\Large{Diplomarbeit \\[38mm]}} % 44mm

	\begin{tabular}{ll}
		\textbf{Verfasser} & Ulrich Belitz, geb. am 4. April 1984\\[3mm]
		\hline
		& \\
		\textbf{Auftraggeber} & SICK AG, Standort Waldkirch\\[2mm]
		\textbf{Betreuer des Auftraggebers} & Dr. Stefan Mack\\[3mm]
		\hline
		& \\
		\textbf{Referent der Hochschule} & Prof. Dr. Christoph Heckenkamp\\[2mm]
		\textbf{Korreferent der Hochschule} & Prof. Dr. Harald Scharfenberg\\[3mm]
		\hline
		& \\
		\textbf{Erstellungszeitraum} & 2. März 2009 bis \today \\[2mm] % alternativ: Bearbeitungszeitraum
		\textbf{Abgabedatum} & 14. September 2009
	\end{tabular}
%	\vfill
\end{center}
%\vfill
\end{titlepage}

\clearpage
\newpage


%\thispagestyle{plain}	% kein Style
%\Large\textbf{Zusammenfassung}\\[5mm]
%\normalsize
\section*{Zusammenfassung}

\clearpage
\newpage

\section*{Danksagung}

\newpage

\section*{Eidesstattliche Erklärung}
Hiermit versichere ich eidesstattlich, dass die vorliegende Diplomarbeit von mir selbstständig und ohne fremde Hilfe angefertigt worden ist, insbesondere, dass ich alle Stellen, die wörtlich oder annähernd wörtlich oder dem Gedanken nach aus Veröffentlichungen, unveröffentlichten Unterlagen und Gesprächen entnommen worden sind, als solche an den entsprechenden Stellen innerhalb der Arbeit kenntlich gemacht habe. Ich bin mir bewusst, dass eine falsche Versicherung rechtliche Folgen haben wird.\\[10mm]
Taunusstein, den \today \\[15mm]
\noindent\rule[1pt]{0.40\textwidth}{1pt}\\
Ulrich Belitz
\newpage

\tableofcontents
\newpage

\listoffigures
\newpage

\listoftables
\newpage

\pagenumbering{arabic}

\section{Anforderungsermittlung}
\subsection{Personen}
\subsubsection{Betreuter}
Der Betreute ist ein körperlich behinderter Mensch, der an Muskelschwäche leidet.
Aus Respekt vor dem körperlich erkrankten Menschen wird in der vorliegenden Ausarbeitung nicht von dem \enquote{Behinderten} gesprochen, sondern stets von dem \enquote{Betreuten}.
\subsubsection{Assistent}
Daas ist ein Typo.


\section{Code-Test}
\begin{lstlisting}
<?php
/* this is a stupid example */
$username = $_POST["username"];
$passwort = $_POST["passwort"];

$pass = md5($passwort);

// another comment
if($username=="Andavos" and
$pass=="fd0d9cdefd5d42dfa36c74a449aa8214") {
   echo "Herzlich Willkommen";
}
else {
   echo "Login Fehlgeschlagen";
}
?>
\end{lstlisting}

\clearpage
\newpage

\bibliographystyle{babplain}
\bibliography{../BibTeX/literatur}
\clearpage
\newpage
\begin{appendix}


\section{Erster Anhang}
\end{appendix}
\end{document}