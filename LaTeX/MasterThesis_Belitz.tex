\documentclass[12pt,a4paper,titlepage]{article} % article scrreprt (KOMA Script Report)

%Kurz-Titel: Assistenzplaner
%Lang-Titel: Assistenzplaner - Serverbasierte Software zur Koordinierung der Assistenten für behinderte Menschen

%\linespread{2.00} % Zeilenabstand
										
%%%%% INPUT %%%%%
\usepackage[utf8]{inputenc}
\usepackage{ngerman}
\usepackage[pdftex]{graphicx} %png und pdf kann eingefügt werden
%\usepackage{graphics}

% für Literatur
\usepackage[ngerman]{babel}
\usepackage{babelbib}

% für \newboolean{allpages} \setboolean{allpages}{false} --> \ifallpages ... \fi
\usepackage{ifthen}

%\usepackage{listings}
%\usepackage{amssymb}
%\usepackage{rotating}
\usepackage{amsmath}
\usepackage{subfigure}
\usepackage{rotating}		%um Tabellen zu drehen

% Header
\usepackage{fancyhdr}
\pagestyle{fancy}
\fancyhf{}	%Löschen der Vorbelegung
\fancyhead[R]{\thepage}
\fancyhead[L]{\leftmark}
%\fancyhead[R]{\leftmark \hspace{10mm} \thepage}
%\fancyfoot[L]{\footnotesize{Nichtlineare Bildsensoren}}
%\fancyfoot[C]{\thepage}
%\fancyfoot[R]{\footnotesize{Ulrich Belitz}}

\title{Diplomarbeit: Nichtlineare Bildsensoren}
\author{Ulrich Belitz}

%Paket für spezielle PDF features.
\usepackage[%
	pdftitle={Diplomarbeit: Nichtlineare Bildsensoren},% Titel des PDF Dokuments.
	pdfauthor={Ulrich Belitz},               % Autor des PDF Dokuments.
	pdfsubject={Nichtlineare Bildsensoren}, % Thema des PDF Dokuments.
	pdfcreator={TeXnicCenter and MikTeX},    % Erzeuger des PDF Dokuments.
	%pdfkeywords={}                          % Schlüsselwörter für das PDF.
	% (Diese werden von Suchmaschinen auch für PDF Dokumente indexiert.)
	pdfpagemode=UseOutlines,   % Inhaltsverzeichnis anzeigen beim Öffnen
	pdfdisplaydoctitle=false,  % Dokumenttitel statt Dateiname anzeigen.
	pdflang=de                 % Sprache des Dokuments.
]{hyperref}

%Anführungszeichen
\usepackage[babel,german=quotes]{csquotes}	%\enquote{TEXT}

%Farben
\usepackage{color}
\definecolor{LinkColor}{rgb}{0, 0, 0}
\definecolor{LinkColor1}{rgb}{0,0,0}
\definecolor{LinkColor2}{rgb}{0,0.5,0}

%Farbeinstellungen für die Links im PDF Dokument.

\hypersetup{%
	colorlinks=true,%        Aktivieren von farbigen Links im Dokument (keine Rahmen)
	linkcolor=LinkColor,%    Farbe festlegen.
	citecolor=LinkColor2,%   	Farbe festlegen.
	filecolor=LinkColor1,%    Farbe festlegen.
	menucolor=LinkColor,%    Farbe festlegen.
	urlcolor=LinkColor1,%     Farbe von URL's im Dokument.
	bookmarksnumbered=true,%  Überschriftsnummerierung im PDF Inhalt anzeigen.
	pdfstartview=FitH%        FitH || FitV  -  Horizontalen Platz ausnutzen
}

% manuelle Silbentrennung
\hyphenation{Knick-stel-le}
\hyphenation{Knick-stel-len}

\newlength{\plotBreite} \setlength{\plotBreite}{0.9\textwidth}

\usepackage{listings,xcolor}
\usepackage{inconsolata}

\definecolor{dkgreen}{rgb}{0,.6,0}
\definecolor{dkblue}{rgb}{0,0,.6}
\definecolor{dkyellow}{cmyk}{0,0,.8,.3}

\lstset{
  language        = php,
  basicstyle      = \small\ttfamily,
  keywordstyle    = \color{dkblue},
  stringstyle     = \color{red},
  identifierstyle = \color{dkgreen},
  commentstyle    = \color{gray},
  emph            =[1]{php},
  emphstyle       =[1]\color{black},
  emph            =[2]{if,and,or,else},
  emphstyle       =[2]\color{dkyellow}}

\begin{document}
\pagenumbering{Roman}
\begin{titlepage}
%\includegraphics{img/sickLogo.pdf}
\hspace{35mm} % per Hand angepasst, muss geändert werden, falls sich Dokumentabmaße ändern!
%\includegraphics{img/LogoFBMN}
\begin{flushright}
	\scriptsize
	Studiengang Optotechnik und Bildverarbeitung\\[4cm]
\end{flushright}
%\vspace*{4cm}
\begin{center}
	
%	zur Erlangung des akademischen Grades eines \\[5mm]
%		\textbf{Diplomingenieur (FH)\\[1cm]}
	\textbf{\Huge{Nichtlineare Bildsensoren}}\\[4mm]%\\[10mm]}
	\textbf{\large{(Non-linear Imagers)}}\\[10mm]
	\textbf{\Large{Diplomarbeit \\[38mm]}} % 44mm

	\begin{tabular}{ll}
		\textbf{Verfasser} & Ulrich Belitz, geb. am 4. April 1984\\[3mm]
		\hline
		& \\
		\textbf{Auftraggeber} & SICK AG, Standort Waldkirch\\[2mm]
		\textbf{Betreuer des Auftraggebers} & Dr. Stefan Mack\\[3mm]
		\hline
		& \\
		\textbf{Referent der Hochschule} & Prof. Dr. Christoph Heckenkamp\\[2mm]
		\textbf{Korreferent der Hochschule} & Prof. Dr. Harald Scharfenberg\\[3mm]
		\hline
		& \\
		\textbf{Erstellungszeitraum} & 2. März 2009 bis \today \\[2mm] % alternativ: Bearbeitungszeitraum
		\textbf{Abgabedatum} & 14. September 2009
	\end{tabular}
%	\vfill
\end{center}
%\vfill
\end{titlepage}

\clearpage
\newpage


%\thispagestyle{plain}	% kein Style
%\Large\textbf{Zusammenfassung}\\[5mm]
%\normalsize
\section*{Zusammenfassung}

\clearpage
\newpage

\section*{Danksagung}

\newpage

\section*{Eidesstattliche Erklärung}
Hiermit versichere ich eidesstattlich, dass die vorliegende Diplomarbeit von mir selbstständig und ohne fremde Hilfe angefertigt worden ist, insbesondere, dass ich alle Stellen, die wörtlich oder annähernd wörtlich oder dem Gedanken nach aus Veröffentlichungen, unveröffentlichten Unterlagen und Gesprächen entnommen worden sind, als solche an den entsprechenden Stellen innerhalb der Arbeit kenntlich gemacht habe. Ich bin mir bewusst, dass eine falsche Versicherung rechtliche Folgen haben wird.\\[10mm]
Taunusstein, den \today \\[15mm]
\noindent\rule[1pt]{0.40\textwidth}{1pt}\\
Ulrich Belitz
\newpage

\tableofcontents
\newpage

\listoffigures
\newpage

\listoftables
\newpage

\pagenumbering{arabic}

\section{Aufgabenstellung} % alternativ: Ziel der Arbeit

\section{Grundlagen}
\subsection{Webentwicklung}
\subsubsection{PHP}
\subsubsection{HTML}
\subsubsection{CSS}
\subsubsection{JavaScript}
-> buch von Koch
\subsubsection{AJAX}
\subsection{Agile Softwareentwicklung}
\subsubsection{User Stories}
\subsection{Modellgetriebene Software-Entwicklung}
\subsubsection{SysML}
\subsection{Datenbanken} % natürlich nur, wenn verwendet ;-)

\section{Anforderungsermittlung / Analyse} %TODO für einen Titel entscheiden
\subsection{Begrifflichkeiten}

\subsubsection{Klient}
Der Klient ist ein körperlich behinderter Mensch, der an Muskelschwäche leidet. %TODO Genauer beschreiben?
Aus Respekt vor dem körperlich erkrankten Menschen wird in der vorliegenden Ausarbeitung nicht von dem \enquote{körperlich Behinderten} gesprochen, sondern stets von dem \enquote{Klienten}.

\subsubsection{Assistent}
Der Assistent ist ein Helfer, der dem Klienten zur Hand geht und ihn im alltäglichen Leben unterstützt. All das, was der Klient nicht selbst erledigen kann, wird durch einen Assistenten erledigt.

\subsubsection{Team}
Das Team besteht aus mehreren Assistenten (während der Entstehung dieser Ausarbeitung aus 7 Stück).

\subsubsection{Monatsplan}
Im Monatsplan sind die Dienst- und Bereitschaftszeiten der Assistenten festgehalten. Für jeden Tag gibt es einen Assistenten der Dienst hat und einen Assistenten der Bereitschaft hat. Während den Bereitschaftszeiten muss der Assistent telefonisch erreichbar sein und innerhalb kurzer Zeit zum Klienten zu kommen und um den Dienst zu übernehmen. Die Dienste gehen frühestens um 13 Uhr, spätestens um 17 Uhr los und enden um 8 Uhr, bzw. 13 Uhr des Folgetages. 

\subsubsection{Stundenkontigent}
Jeder Assistent arbeit unterschiedlich viel. Die Monatsstundenzahl variiert zwischen 30 und 130. Jeder Assistent möchte sein Stundenkontigent möglichst gut ausgeschöpft bekommen. Bei der Dienstplanerstellung ist darauf zu achten, dass alle Assistenten möglichst gleichmäßig ihr Stundenkontigent ausgeschöpft bekommen. %TODO die letzten zwei Sätze direkt hintereinadner hören sich doof an!

\subsection{Ist-Prozess}
Vor Beginn der Arbeiten des Autors war der Prozess zur Dienstplanerstellung komplett manuell. Der Klient hat für die Dienstplanerstellung eine E-Mail mit Hinweisen für den Monat an sein Team geschickt und darum gebeten Rückmeldung zu geben, wie die Verfügbarkeiten sind.
Die Assistenten haben auf diese E-Mail geantwortet und darin beschrieben wann sie Zeit für Dienste haben.
Der Klient hat manuell mit Hilfe einer Excel-Tabelle den Dienstplan erstellt und musste dabei ständig zwischen Excel-Tabelle und allen Antworten des Teams (sieben an der Zahl) wechseln. Nach der Fertigstellung des Dienstplans hat der Klient eine E-Mail mit dem Dienstplan an sein Team geschickt.
Die Dienstplanerstellung hat den Klienten viel Zeit (ca. vier bis fünf Stunden im Monat) und Nerven gekostet.
%TODO Ablaufdiagramm?


\subsection{Soll-Prozess}
%TODO ein möglicher Soll-Prozess - es gibt mehrere Möglichkeiten
Der Klient gibt auf einer Webseite die Dienstzeiten und Kommentare für den kommdenen Monat ein. Per Knopfdruck wird das Team per E-Mail informiert und darum gebeten die freien Tage auf einer Webseite (Link ist in der E-Mail enthalten) einzugeben. Sobald alle Assistenten ihre Eingaben getätigt haben wird ein Dienstplan-Vorschlag erstellt und dem Klienten eine E-Mail geschickt mit einem Link auf den Dienstplan. Der Dienstplan kann von dem Klienten manuell editiert werden und nach den eigenen Vorlieben (die bei der automatischen Erstellung schon größtenteils berücksichtigt wurden) anpassen. Der Klient gibt den Dienstplan frei und das Team wird automatisch per E-Mail darüber benachrichtig, dass der Dienstplan verfügbar ist.
Der Zeitaufwand für die Assistenten bleibt unverändert. Der Zeitaufwand für den Klient sinkt auf unter eine halbe Stunde.
%TODO Ablaufdiagramm?

\subsection{User Stories}
Aus der Formulierung des Soll-Prozesses sind einige User-Stories abgeleitet worden:
%TODO User-Stories einfügen

\subsubsection{Klient}
\begin{itemize}
\item Als Klient möchte ich für den nächsten Monat angeben können, wie die Dienstzeiten sind.
\item Als Klient möchte ich die Möglichkeit haben, den Dienstplan nach meinen Wünschen anzupassen.
\item Als Klient möchte ich eine Möglichkeit haben mein Team zu verwalten.
\item Als Klient möchte ich die Assistenten bewerten, um guten Assistenten bei der Einteilung Vorrang zu gewähren.
\item Als Klient möchte ich bei den Assistenten Vorlieben für Wochentage eingeben können.
\item Als Klient möchte ich bei der manuellen Nachbearbeitung des Dienstplanes die Stundenverteilung sehen können.
\item Als Klient möchte ich Notizen zum Tag erstellen können, die für jeden sichtbar sind (öffentliche Notizen).
\item Als Klient möchte ich Notizen zum Tag erstellen können, die nur für mich sichtbar sind (private Notizen).
\item Als Klient möchte ich mehrere Vorschläge für einen Dienstplan bekommen, die ich anpassen kann.
\item Als Klient möchte ich eine Mail an mein Team verschicken können, mit der Bitte freie Termine einzutragen.
\item Als Klient möchte ich Hinweise zum Dienstplan eingeben können.
\end{itemize}

\subsubsection{Assistent}
\begin{itemize}
\item Als Assistent möchte ich angeben können, wann ich arbeiten kann.
\item Als Assistent möchte ich mein Stundenkontingent möglichst gut ausgeschöpft bekommen.
\item Als Assistent möchte ich meine Eingaben im Kalender ändern können.
\item Als Assistent möchte ich sehen, wann ich meine Dienste habe.
\item Als Assistent möchte ich sehen, wann ich Bereitschaft habe.
\item Als Assistent möchte ich eine Möglichkeit haben Termine zu favorisieren.
\item Als Assistent möchte ich bei der Angabe von Terminen auch Bemerkungen machen können, um weitere Informationen transportieren zu können.
\end{itemize}

\subsubsection{Allgemein}
\begin{itemize}
\item Als Anwender möchte ich sehen, wann der Dienstplan zum letzten Mal geändert wurde.
\item Als Anwender möchte ich mich mit einem eigenen Account beim System anmelden können, um nur die Dinge zu sehen, die mich interessieren und nicht aus Versehen ungewollte Änderungen zu machen.
\end{itemize}


\subsection{Abgeleitete Anforderungen}
\subsubsection{Serverbasierte Anwendung}
\subsubsection{User Management}


\section{Software-Design} % oder reinfach nur Design?
Modellgetriebene Software-Entwicklung
SysML Modell
UML Ablaufdiagramme
Datenbankmodell?

\section{Implementierung}
\subsection{Wahl der Programmiersprachen}
\subsection{Entwicklung des Algorithmus}
von ganz dumm zu schlau


\subsection{Code-Listing-Test}
\begin{lstlisting}
<?php
/* this is a stupid example */
$username = $_POST["username"];
$passwort = $_POST["passwort"];

$pass = md5($passwort);

// another comment
if($username=="Andavos" and
$pass=="fd0d9cdefd5d42dfa36c74a449aa8214") {
   echo "Herzlich Willkommen";
}
else {
   echo "Login Fehlgeschlagen";
}
?>
\end{lstlisting}

\clearpage
\newpage

\section{Irgendwo...}
- "denglische" Formulierungen wie ausgeloggt werden vermieden

\section{Fazit und Ausblick} % alternativ: Diskussion und Ausblick

\bibliographystyle{babplain}
\bibliography{../BibTeX/literatur}
\clearpage
\newpage
\begin{appendix}


\section{Entwicklungsumgebung}
\subsection{Hardware}
\subsection{XAMPP}
\subsection{PhpStorm}
\subsection{GitHub Repository und Client}
\end{appendix}
\end{document}